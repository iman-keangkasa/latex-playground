\documentclass[a4paper,10pt]{article}
\title{Probabilistically Voxelated Occupancy Map for Path-Planning}
\author{}
\usepackage{textcomp}
\usepackage{cite}
\usepackage[colon]{natbib}

\begin{document}
\maketitle

\section{Introduction}
Map-building is one half of a solution to simultaneous localization and mapping
(SLAM) problem. SLAM acts as a utility to solve robotic exploration in an unknown
environment. SLAM is stochastic and involves error in data reading. Stochastic
information from multiple sensors are used to form information gain through
filtering and data fusion. 

Map building requires the same data fusion for local
map registration. The registration method involves filtering technique to
transform a local map into a global map. However, most of the solution for three
dimensional map-building are in the form of point clouds, occupancy grid map,
and a variant of occupancy grid map. Point clouds involves
rich representation of geometric data for autonomous robots. Unfortunately, point
clouds increases computational
load \citep{Borrmann2008,Cole2006,Engelhard2011,Weingarten2005}. These point clouds are discrete series of metrical data. A
map suiting the robotic ability to move in three dimensional space requires
three dimensional map that emphasize the geometric information of an
environment and the have a structure that will not hinder online computation.

In tandem to automatic motion, this map should represent the probabilistic
nature of the sensor and the locomotion of the robot to be effectively used in
SLAM solution. \citet{Hornung2013} introduced a three dimensional occupancy grid
map, coined as Octomap, that extend the two dimensional grid map introduced by \citet{Elfes1989}.

Octomap uses grid cells seeded with binary values to represent occupancy in
space. The grid cells are stacked continuosly in series of cubic primitives.
Octomap thus far are used only to represent static environments. This paper will
attempt to create a variant of octomap, continous-octomap, to discern its
feasibility on dynamic environment representation for path-planning purposes. 

\section{Related Works on Evolution of Map Building}
A confident map is instrumental for a correct localization. Since mid-1990's, the
mobility of mobile robots have increased from three degree-of-freedoms to six
degree-of-freedoms (DOFs). A manipulator such as an articulated  industrial
robot arm also have 6-DOF end effector. An autonomous robot with these 
mobilities requires confident map that represent an environment in three 
dimensional space. Three types of 
maps commonly used for autonomous robots are point clouds, graph maps, and
occupancy grid maps. 

Point clouds are three dimensional representation of spatial space. 
Point clouds are
often used with iterative closest point matching (ICP) or scan matching to
register local maps. These type of maps are limited by their large number of
information containing metrical data that puts computational strain to an
algorithm \citep{Olson2009}. However, with the advent of parellel computing,
and the optimization of a library called OpenCL \citep{Rusu2011}, the
computational efficiency for point clouds manipulation may increase.

Graph maps use graph theory to represent an environment semantically. The
concept of graph maps relates poses and features as nodes and abstracting raw
sensor data into pose-graph \citep{Grisetti}. A pose of a robot contains
positional information that reflects the degree of freedom of the robot.
Unlike point clouds, graph
maps lacks geometric structure and oftenly used for optimizing localization
processes in SLAM solution.

Two dimensional occupancy grid map is a type of map that contains probability description of an
environment from sensor reading. The map uses Poisson distribution to ascertain
the occupancy of a two dimensional space. This map was introduced by \citet{Moravec1988} and
\citet{Elfes1989} using sonar sensors, as an output, to map an environment for
an autonomous robot. From this map, the environment is divided into planar grid
cell to represent occupancy. The concept of occupancy grid cell is still used
today to develop a more advance mapping and localization method particularly for
SLAM solution \citep{Ray,Birk}. 

Three dimensional occupancy grid map such as Octomap \citep{Hornung2013} enables
robot to move in three-dimensional space with three to six DOFs. Octomap extend
the use of grid map-building into three dimensional space and embed each grid
cell with a relaxed logit function to give a binary value of occupancy. In this
map, occupied space is represented by a probability value of "1" and unoccupied
space is represented by a probability value of "0". Octomap adopts octree data
structure to allow efficient computation of the localization posterior
distribution. However, currently, octomap is only being used for semi-online
autonomous robot. With a semi-online design, an autonomous robot perform poorly
in highly dynamic environments. 

A part of SLAM solutions is on
registering multiple maps from local coordinates into maps in global coordinate
system. As autonomous robot moves in an unknown environment, a collection of
maps in the form of scans are introduced to its vision system. One method of
map registration is done post-exploration \citep{Burgard1999}. However, online
registration is more desirable in the context of an autonomous robot. A
semi-online map registration would bundle a collection of map before registering
the map globally during exploration. 

This paper extend the definition of occupancy in octomap by introducing
continous probability value instead of binary values in each grid cell. 
We will compare
the perfomance of a number of path-planning algorithms in these maps. We will use octomap as the
benchmark. The metric of perfomance is based on time to complete a path planning
from a predefined starting position to a final position in a static environment.
This approach is reapplied for environment containing moving objects and human.
\bibliography{../bibtex/masters_thesis}
\bibliographystyle{apalike}
\end{document}
