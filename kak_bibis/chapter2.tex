\documentclass[a4paper, 10pt]{article}
\title{Chapter Two: The Risk and The Safety in Aviation Industry}
\author{Rabiah binti Tukiman}
\usepackage{textcomp}
\begin{document}
\maketitle
\section{Introduction}
Nowadays, the air transport system is recognized as one of
the fastest growing areas within the transport sector as well as
in overall regional and world economies. According to many
forecasts this growth will continue at an average rate of 5\% in
passenger and 6\% in freight transport demand over the next
two decades. It will primarily be driven by overall economic
growth, further globalization of the regional and world’s
economy, and even further decreasing of airfares thanks to
among other factors the growth of the low-cost carrier’s market
share. The system infrastructure – airports and Air Traffic
Control/Management (ATC/ATM) although in many cases
acting as temporal “bottlenecks” are expected to be able to
support such growth safely, efficiently and effectively.\par

Physically and operationally, the air transport system is a
rather complex system with the main components; the airlines, the
airports and the air traffic control services, interacting with each
other on different hierarchical levels constituting a very complicated, 
highly distributed network of human operators,
procedures and technical/technological systems. In particular,
risk of accidents and related safety in such a complex system is
crucially influenced by interactions between the various
components and elements. This implies that providing a
satisfactory level of safety (i.e., low risk of accident) is more
than making sure that each of the components and elements
functions safely. Due to such inherent complexity and
severe consequences of accidents, risk and safety have always
been considered as issues of the greatest importance for the
contemporary air transport system. Consequently, they have
been a matter of continuous research from different aspects and
perspectives ranging from the purely technical/technological to
the strictly institutional. In general, the former have dealt with
design of safe aircraft and other system facilities and
equipment. The later have implied setting up adequate
regulations for system design and operations.
\section{Aviation Risk and Safety}
For a long time, risk and safety have been differently and
ambiguously interpreted depending on the system and purpose. 
For technical systems, risk is related to the chance of
failure of components or of the entire system causing exposure
to hazard and related consequences. In economic business
systems, risk is a chance of being exposed to the hazard of
losing business opportunities and/or money due to making
decisions under uncertain circumstances. In social systems, risk
is the chance of being exposed to the hazard of injuries and/or
losing of life. Consequently, risk could be considered as
combination of the probability (or frequency of occurrence)
and the magnitude of consequences (or severity) of a hazardous
event.\par

In the air transport system, risk and safety have always been
related to air traffic accidents which resulted in the significant
loss of life and property (aircraft and the property on the
ground). Assuming that making an air trip is an individual
choice and that the system deploys some resources to satisfy
such choice, four types of risks can be identified in the air
transport system:
\begin{enumerate}
		\item real risk to an individual (determined onthe basis of future circumstances after their full development,
				frequently incorporated in decisions on introduction of new
				aerospace technologies in any system component); 
		\item statistical risk of occurrence of an accident (important for companies
				providing insurance, determined by the available statistical data
				on the incidents and accidents); 
		\item predicted risk (important	for air transport authorities while introducing changes in
				technologies and air traffic patterns, determined from
				methodologies using some relevant historical research); and 
		\item perceived risk (important for users of the air transport system
				and determined by the individual’s intuition, feeling and
				perception).
\end{enumerate}
\par

In addition, air traffic accidents may have some features
distinguishing them from accidents in other transport modes as
follows: 
\begin{enumerate}
		\item they may occur at any point in time and space
mainly because flights may take place over large areas; 
		\item the
primal target groups exposed to the risk exposure are
passengers and crew; in addition, individuals on the ground
may be exposed but generally have a lower probability of
losing life or property; 
		\item they are relatively rare events but
usually with severe consequences; 
		\item conditionally, each of
them can be classified as an inherently risky although highly
unlikely (but still possible) event; and 
		\item risk of an accident is inherently present during the flight.
\end{enumerate}
\par

Risk implies exposure of an individual to the hazard of an
air traffic accidental event (collision between aircraft, and/or
collision between the aircraft and terrain). This could result in
losing life or getting severe injuries both onboard the aircraft
and/or on the ground, damaging and/or destroying property (the
aircraft and eventually buildings on the ground), and
contamination of the environment (water and soil) by burning
and/or leaking fuel and oil, and hazardous cargo.\par

In the above-mentioned context, assessing the risk of
occurrence of an air traffic accident with the associated
consequences can be used as a measure of the system safety for
people, systems and environment.\par

\section{Standards And Recommended Practices In Aviation Industry as a Non-profit Risk Management}
Standards And Recommended Practices (SARPs) are technical specifications adopted by the International Civil Aviation Organization (ICAO) in accordance with Article 37 of the Convention on International Civil Aviation in order to achieve "the highest practicable degree of uniformity in regulations, standards, procedures and organization in relation to aircraft, personnel, airways and auxiliary services in all matters in which such uniformity will facilitate and improve air navigation".\par

SARPs are published by ICAO in the form of Annexes to Chicago Convention. SARPs do not have the same legal binding force as the Convention itself, because Annexes are not international treaties. Moreover States agreed to "undertake to collaborate in securing (...) uniformity", not to "comply with". Each Contracting State may notify the ICAO Council of differences between SARPs and its own regulations and practices. Those differences are published in the form of Supplements to Annexes.\par

A Standard is defined by ICAO as "any specification for physical characteristics, configuration, material, performance, personnel or procedure, the uniform application of which is recognized as necessary for the safety or regularity of international air navigation and to which Contracting States will conform in accordance with the Convention".\par

A Recommended Practice is defined by ICAO as "any specification for physical characteristics, configuration, material, performance, personnel or procedure, the uniform application of which is recognized as desirable in the interest of safety, regularity or efficiency of international air navigation and to which Contracting States will endeavour to conform in accordance with the Convention".\par

\subsection{Aviation Safety Implementation and Investigatory Bodies}
The aviation safety implementations exist in all countries worldwide. However the investigatory bodies exist in most of the countries in the world. These bodies address any accidents and incidents related to aviation and reports their findings.\par

In Australia, the Australian Transport Safety Bureau is the federal government body responsible for investigating transport-related accidents and incidents, covering air, sea, and rail travel. Formerly an agency of the Department of Infrastructure, Transport, Regional Development and Local Government, in 2010, in the interests of keeping its independence it became a stand-alone agency.\par
In Brazil, the Aeronautical Accidents Investigation and Prevention Center (CENIPA) was established under the auspices of the Aeronautical Accident Investigation and Prevention Center, a Military Organization of the Brazilian Air Force (FAB). The organization is responsible for the activities of aircraft accident prevention, and investigation of civil and military aviation occurrences. Formed in 1971, and in accordance with international standards, CENIPA represented a new philosophy: investigations are conducted with the sole purpose of promoting the "prevention of aeronautical accidents".\par

In Canada, the Transportation Safety Board of Canada (TSB), is an independent agency responsible for the advancement of transportation safety through the investigation and reporting of accident and incident occurrences in all prevalent Canadian modes of transportation — marine, air, rail and pipeline.\par

In France, the agency responsible for investigation of civilian air crashes is the Bureau d'Enquêtes et d'Analyses pour la Sécurité de l'Aviation Civile (BEA). Its purpose is to establish the circumstances and causes of the accident and to make recommendations for their future avoidance.\par

In Germany, the agency for investigating air crashes is the Federal Bureau of Aircraft Accidents Investigation (BFU). It is an agency of the Federal Ministry of Transport and Digital Infrastructure. The focus of the BFU is to improve safety by determining the causes of accidents and serious incidents and making safety recommendations to prevent recurrence.\par

In Hong Kong, the Civil Aviation Department's Flight Standards \& Airworthiness Division and Accident Investigation Division are charged with accident investigation involving aircraft within Hong Kong.\par

Until May 30, 2012, the Directorate General of Civil Aviation investigated incidents involving aircraft. Since then, the Aircraft Accident Investigation Bureau has taken over investigation responsibilities.\par

In Indonesia, The National Transportation Safety Committee (NTSC Or KNKT) is responsible for the investigation of incidents and accidents, including air accidents. Its aim is the improvement of safety in Indonesia.\par

In Italy, the Agenzia Nazionale per la Sicurezza del Volo (ANSV), has two main tasks: conducting technical investigations for civil aviation aircraft accidents and incidents, while issuing safety recommendations as appropriate; and conducting studies and surveys aimed at increasing flight safety. The organization is also responsible for establishing and maintaining the "voluntary reporting system." Although not under the supervision of the Ministry of Infrastructure and Transport, the ANSV is a public authority under the oversight of the Presidency of the Council of Ministers of Italy.\par

The Japan Transport Safety Board investigates aviation accidents and incidents. The Aircraft Accident Investigation Commission investigated aviation accidents and incidents in Japan until October 1, 2001, when the Aircraft and Railway Accidents Investigation Commission (ARAIC) replaced it, and the ARAIC did this function until October 1, 2008, when it merged into the JTSB.\par

In Mexico the Directorate General of Civil Aviation (DGAC) investigates aviation accidents.\par

In The Netherlands, the Dutch Safety Board (Onderzoeksraad voor Veiligheid) is responsible for the investigation of incidents and accidents, including air accidents. Its aim is the improvement of safety in The Netherlands. Its main focus is on those situations in which civilians are dependent on the government, companies or organizations for their safety. The Board solely investigates when incidents or accidents occur and aims to draw lessons from the results of these investigations. The Safety Board is objective, impartial and independent in its judgment. The Board will always be critical towards all parties concerned.\par

In New Zealand, the Transport Accident Investigation Commission (TAIC), is responsible for the investigation of air accidents. "The Commission's purpose, as set out in its Act, is to determine the circumstances and causes of aviation, rail and maritime accidents, and incidents, with a view to avoiding similar occurrences in the future, rather than to ascribe blame to any person." The TAIC will investigate in accordance with annex 13 of the ICAO. \par

In Russia, the Interstate Aviation Committee (IAC, MAK according to the original Russian name) is an executive body overseeing the use and management of civil aviation in the Commonwealth of Independent States. This Organization investigates air accidents in the former USSR area under the umbrella of the Air Accident Investigation Commission of the Interstate Aviation Committee.\par

In Taiwan, the Aviation Safety Council (ASC) is the independent government agency that is responsible for aviation accident investigations. Established in 1998, ASC is under the administration of the Executive Yuan and independent from Civil Aeronautics Administration of Taiwan. The ASC consisted of five to seven board members, including a chairman and a vice chairman, appointed by the Premier. The managing director of ASC manages the day-to-day function of the organization, including accident investigations.\par

In the United Kingdom, the agency responsible for investigation of civilian air crashes is the Air Accidents Investigation Branch (AAIB) of the Department for Transport. Its purpose is to establish the circumstances and causes of the accident and to make recommendations for their future avoidance.\par

United States civil aviation incidents are investigated by the National Transportation Safety Board (NTSB). NTSB officials piece together evidence from the crash site to determine likely cause, or causes. The NTSB also investigates overseas incidents involving US-registered aircraft, in collaboration with local investigative authorities, especially when there is significant loss of American lives, or when the involved aircraft is American built.\par
\section{Implementation of Risk Management System at a Corporate Level}
\subsection{The Definitions in Risk Management in Context of a Corporate Governance}%lit3_chapter2 page 30
\begin{description}
		\item[Hazard] A hazard is defined as a condition or an object with the potential to cause injuries to person-
				nel, damage to equipment or structures, loss of material, or reduction of ability to perform a
				prescribed function. The description of the potential outcome of the hazard is
				the consequence.
		\item[Hazard Identification] Hazard identification (HAZID) is the process of identifying hazards, which forms the essen-
				tial first step of a risk assessment. There are two possible purposes in identify-
				ing hazards: to obtain a list of hazards for subsequent evaluation using other risk assessment
				techniques (failure case selection) or to perform a qualitative evaluation of the significance of
				the hazards and the measures for reducing the risks from them (hazard assessment).
		\item[Risk] Risks are unforeseen deviations from expected values caused by accidental interferences de-
				riving from the unpredictability of the future. The ratio between
				the probability of occurrence and the expected measure of damages is referred to as individual
				risk. Besides the negative implication of risk, risk management
				is always a balancing act between risk opportunities and threats.
\end{document}
