\documentclass[a4paper, 10pt]{article}
\title{Two dimensional tracking and labeling for three dimensional point cloud labeling}
\usepackage{natbib}
\usepackage{soul,color}
%\usepackage[dvipsnames]{xcolor}
%defining highlight color
\begin{document}
\maketitle
%\begin{document}
\section{Abstract}
A database for an isolated three dimensional point clouds has yet been introduced. 
The construction of this database can reduce the time of training for neural network
model on human motion because isolated point cloud would not require the analysis and 
the computation of the whole scene in three dimensional space. 
This paper will attempt to label three dimensional point clouds with training
in two dimensional space.
Human identification is done followed by tracking on the sequences of images from
RGB-D output. Classification of the human motion is done around the boundaries
of the tracking. These classification consist of five labels: walking, running,
ducking, falling, and jumping. The training for these classifier are done using
various 2D image databases. The two dimensional tracking boundaries are
projected into three dimensional space that subsequently discriminate the point
clouds to its corresponding label. These labels and the subset point clouds are
used to construct a human action point cloud database. 

\section{Introduction}
%\section{Point cloud databases on Human movement}
%\section{Methodology}
%\section{Movement taxonomy}
%\section{Data structure on a human reactive action database}

\end{document}
