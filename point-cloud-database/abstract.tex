\documentclass[a4paper, 10pt]{article}
\title{Three Dimensional Point Cloud Database of Human Action while Moving/Ambulating}
\usepackage{natbib}
\usepackage{soul,color}
%\usepackage[dvipsnames]{xcolor}
%defining highlight color
\begin{document}
\maketitle
%\begin{document}
\section{Introduction}

\hl{Point cloud database with labels involving reactive action in 
hazardous environment does not exist} (Review is still on going).
This paper will attempt to delineate a method to extract point clouds from RGB-D ouput 
to consturct a dataset with labelled and isolated point clouds. With the construction 
of this dataset, we will provide focused taxonomy 
on the human action involving ambulation. \hl{We will further define the taxonomy for 
ambulation on alerted action. In this context, alerted action is a reactive response 
in a form of human movement in hazardous environment} (Further database improvement). With this database, computational expense and computation time for neural network training 
can be decreased. 

\section{Point cloud databases on Human movement}
\section{Methodology}
\section{Movement taxonomy}
\section{Data structure on a human reactive action database}

\end{document}
