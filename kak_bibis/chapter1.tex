\documentclass[a4paper,10pt]{article}
%defining the title
\title{Chapter One: The Insight, The Accidents and The Incidents That Define The Aviation Industry}
\author{Rabiah binti Tukiman}
\usepackage{textcomp}
\begin{document}
\maketitle

%to add newline \\
%to add new paragraph \par
%to add subsection \subsection

\section{The Definition of Accidents and Incidents}

In reference to the \textbf{Convention on International Civil Aviation, \textit{Annex 13}}, an \textit{aviation accident} is defined as an occurance 
associated with the operation of an aircraft, which takes place from the time any person boards the aircraft with the intention of flight 
until all such persons have disembarked, and in which 
\textbf{(a)} a person is fatally or seriously injured, 
\textbf{(b)} the aircraft sustains significant damage or structural failure, or 
\textbf{(c)} the aircraft goes missing or becomes completely inaccessible. 
I refer to the same \textit{Annex 13} to define an \textit{aviation incident} as, other than an accident
the association of the operation of an aircraft that affects or could affect the safety of avionic operation. \par
	I also define in the context of this paper the term aviation industry as the business sector dedicated to manufacturing and operating all types of aircraft. I refer the air traffic controllers as the constituent that enables the operation of aviation industry aiming to cease any aviation accidents per definition by \textit{Annex 13} and to cease any aviation incident.
\section{The Causes of Aviation Accidents}
Causes affecting the accident rate in air transportation can classified from various factors and points of
view. The most general and probably the most transparent way of classification is dependence on
human action or failure, technical and meteroroligcal factors. As far as the organizational or
legislative shortcomings are concerned, they could also be instrumental in supporting the factors
mentioned , mostly as a result of poor adherence to legislative procedures or mismanagemnt of air
operation.\par
There is a range of causes to air accidents. In view of the fast development taking place in almost all
the field of aviations, the occurence of air accidents caused by aviation technology is reducing. The
development, however, is adding to the complexity of systems and raising the level of seriousness, all
that to be managed by the aircrew. This very cause appears to be increasing in direct prorportion to
the accidents caused by human factors. Currently its contribution is at about 80\%.\par 	

The category of most serious air accidents is made up of the so called: CFIT (Controlled Flight
Into Terrain). It involves accidents at which the aircaft is flying on a final approach course for landing
at poor visibility or when flying in clouds, or even by night. Despite of the perfect status of the
airctraft, it hits the ground. The fact that high-capacity aircraft are most involved in them results in
very high numbers of casualties, as a rule.\par

Another category of serious accidents is loss of control over the aircraft (against caused by human
factor). With the majority of accidents, more than one factor or cause is involved. There often comes
to the so-called cumulation of errors, whereas the single errors by themselves appear to bear almost
no importance (statistics prove that at every flight of an aircraft the comes to an occurrence of
errors, at an average of 1,9 attributed to the aircrew. As soon as the aircrew by taking appropriate
measures fails to eliminate or stop further negative development of events, control over the aircraft
is doomed to be lost.\par

The cause to failures and the follow-up loss of control over the aircraft can be attributed to the
incompetence of the aircrew (lack of experiences, insufficient training or errors in the cooperation,
lack of attentiveness, misunderstandings etc.).\par

It is important to point out that at air accidents are not only monitored in terms of their direct
causes, but also in terms of the phase of flight when they occur.It has been found out that as much as
50\% of all the accidents took place during the approach to landing, which represents only 4\% of the
total flight time. Another 27\% of accidents occured during takeoffs and initial climbs representing
only some 2\% of the flight time. A simple addition of the percentages reveas that more than
of all air accidents occur wtihin a relatively short legs of flight.\par

I believe there are five defining moments that shaped the aviation regulation to date. These aviation disasters had impacted, redefine, and reshape the aviation regulation to a standard that a redundant and excessive. These four disasters are 
\textbf{(a)} the September 11 disasters in 2011, 
\textbf{(b)} the Tenerife airport disaster in 1977, 
\textbf{(c)} the JAL Flight 123 of 1985,
\textbf{(d)} the Charkhi Dadri mid-air collision in 1996, and
\textbf{(e)} the missing flight MH370.

\subsection{The September 11 disaster in 2011}
The September 11 attacks,also referred to as 9/11, were a series of four coordinated terrorist attacks by the Islamic terrorist group al-Qaeda against the United States on the morning of Tuesday, 
September 11, 2001. 

The attacks killed 2,996 people, injured over 6,000 others, and caused at least \$10 billion in infrastructure and property damage. 
Additional people died of 9/11-related cancer and respiratory diseases in the months and years following the attacks. \par
Four passenger airliners operated by two major U.S. passenger air carriers, The United Airlines and the American Airlines, all of which departed from airports in the northeastern 
part of the United States bound for California—were hijacked by 19 al-Qaeda terrorists. 

Two of the planes, American Airlines Flight 11 and United Airlines Flight 175, were crashed into the North and South towers, respectively, of the World Trade Center complex in Lower Manhattan. Within an hour and 42 minutes, both 110-story towers collapsed. Debris and the resulting fires caused a partial or complete collapse of all other buildings in the World Trade Center complex, including the 47-story 7 World Trade Center tower, as well as significant damage to ten other large surrounding structures. A third plane, American Airlines Flight 77, was crashed into the Pentagon (the headquarters of the U.S. Department of Defense) in Arlington County, Virginia, which led to a partial collapse of the building's west side. The fourth plane, United Airlines Flight 93, was initially flown toward Washington, D.C., but crashed into a field in Stonycreek Township near Shanksville, Pennsylvania, after its passengers thwarted the hijackers.\par

9/11 is the single deadliest terrorist attack in human history and the single deadliest incident for firefighters and law enforcement officers in the history of the United States, with 343 and 72 killed, respectively. Different from the other aviation disasters, the accidents involved motives. \par

I explain the chronology of the 9/11 disaster.  At 8:46 am, five hijackers crashed American Airlines Flight 11 into the northern façade of the World Trade Center's North Tower (1 WTC). At 9:03 am, another five hijackers crashed United Airlines Flight 175 into the southern façade of the South Tower. Five hijackers flew American Airlines Flight 77 into the Pentagon at 9:37 am. A fourth flight, United Airlines Flight 93, crashed near Shanksville, Pennsylvania, southeast of Pittsburgh, at 10:03 a.m. after the passengers fought the four hijackers. Flight 93's target is believed to have been either the Capitol or the White House. Flight 93's cockpit voice recorder revealed crew and passengers tried to seize control of the plane from the hijackers after learning through phone calls that Flights 11, 77, and 175 had been crashed into buildings that morning. Once it became evident that the passengers might gain control, the hijackers rolled the plane and intentionally crashed it.
Collapse of the towers as seen from across the Hudson River in New Jersey, the north face of Two World Trade Center (south tower), immediately after being struck by United Airlines Flight 175 \par

\subsection{The Tenerife airport disaster in 1977}
On March 27, 1977, two Boeing 747 passenger jets, KLM Flight 4805 and Pan Am Flight 1736, collided on the runway at Los Rodeos Airport (now Tenerife North Airport), on the Spanish island of Tenerife, Canary Islands, killing 583 people, making it the deadliest accident in aviation history.\par

A terrorist incident at Gran Canaria Airport had caused many flights to be diverted to Los Rodeos, including the two aircraft involved in the accident. The airport quickly became congested with parked airplanes blocking the only taxiway and forcing departing aircraft to taxi on the runway instead. Patches of thick fog were drifting across the airfield, so that the aircraft and control tower were unable to see one another.\par

The collision occurred when KLM 4805 initiated its takeoff run while Pan Am 1736, shrouded in fog, was still on the runway and about to turn off onto the taxiway. The impact and resulting fire killed everyone on board the KLM plane and most of the occupants of the Pan Am plane, with only 61 survivors in the front section of the aircraft.\par

The subsequent investigation by Spanish authorities concluded that the primary cause of the accident was the KLM captain's decision to take off in the mistaken belief that a takeoff clearance from air traffic control (ATC) had been issued. Dutch investigators placed a greater emphasis on mutual misunderstanding in radio communications between the KLM crew and ATC, but ultimately KLM admitted that their crew was responsible for the accident and the airline agreed to financially compensate the relatives of all of the victims.\par

The disaster had a lasting influence on the industry, highlighting in particular the vital importance of using standardized phraseology in radio communications. Cockpit procedures were also reviewed, contributing to the establishment of crew resource management as a fundamental part of airline pilots' training.\par

As a consequence of the accident, sweeping changes were made to international airline regulations and to aircraft. Aviation authorities around the world introduced requirements for standard phrases and a greater emphasis on English as a common working language.\par

Air traffic instruction should not be acknowledged solely with a colloquial phrase such as "OK" or even "Roger" (which simply means the last transmission was received), but with a readback of the key parts of the instruction, to show mutual understanding. The phrase "take off" is now spoken only when the actual takeoff clearance is given or when cancelling that same clearance (i.e. "cleared for take-off" or "cancel take-off clearance"). Up until that point, aircrew and controllers should use the phrase "departure" in its place, e.g. "ready for departure". Additionally, an ATC clearance given to an aircraft already lined-up on the runway must be prefixed with the instruction "hold position".\par

Cockpit procedures were also changed. Hierarchical relations among crew members were played down. More emphasis was placed on team decision-making by mutual agreement. Less experienced flight crew members were encouraged to challenge their captains when they believed something was not correct, and captains were instructed to listen to their crew and evaluate all decisions in light of crew concerns. This concept was later expanded into what is known today as crew resource management (CRM), training which is now mandatory for all airline pilots.\par
In 1978, a second airport on the island was opened: the new Tenerife–South Airport (TFS). This airport now serves the majority of international tourist flights
. Los Rodeos, renamed to Tenerife North Airport (TFN), was then used only for domestic and inter-island flights. In 2002, a new terminal was opened and Tenerife North once again carries international traffic, including budget airlines. The Spanish government installed a ground radar at Tenerife North Airport following the accident.

\subsection{The JAL Flight 123 in 1985}
Airlines Flight 123 was a scheduled domestic Japan Airlines passenger flight from Tokyo's Haneda Airport to Osaka International Airport, Japan. On Monday, August 12, 1985, a Boeing 747SR operating this route suffered a sudden decompression twelve minutes into the flight and crashed in the area of Mount Takamagahara, Ueno, Gunma Prefecture, 100 kilometres (62 miles) from Tokyo thirty-two minutes later. The crash site was on Osutaka Ridge, near Mount Osutaka.\par

Japan's Aircraft Accident Investigation Commission officially concluded that the rapid decompression was caused by a faulty repair by Boeing technicians after a tailstrike incident during a landing at Osaka Airport seven years earlier. A doubler plate on the rear bulkhead of the plane had been improperly repaired, compromising the plane's airworthiness. Cabin pressurization continued to expand and contract the improperly repaired bulkhead until the day of the accident, when the faulty repair finally failed, causing the rapid decompression that ripped off a large portion of the tail and caused the loss of hydraulic controls to the entire plane.\par

The aircraft, configured with increased economy class seating, was carrying 524 people. Casualties of the crash included all 15 crew members and 505 of the 509 passengers; some passengers survived the initial crash but subsequently died of their injuries hours later, mostly due to the Japan Self-Defense Forces’s decision to wait until the next day to go to the crash site, after denying an offer from a nearby United States Air Force base to start an immediate rescue operation. It remains the deadliest single-aircraft accident in aviation history, the second-deadliest Boeing 747 accident and the second-deadliest aviation accident after the collision of two Boeing 747s in the 1977 Tenerife airport disaster.\par

I will delineate the chronology of the events that led to the crash of JAL Flight 123. The aircraft landed at Haneda from New Chitose Airport at 4:50PM as JL514. After more than an hour on the ramp, Flight 123 pushed back from gate 18 at 6:04 p.m. and took off from Runway 15L at Haneda Airport in Ōta, Tokyo, Japan, at 6:12 p.m., twelve minutes behind schedule. About 12 minutes after takeoff, at near cruising altitude over Sagami Bay, the aircraft's aft pressure bulkhead burst open due to a pre-existing defect stemming from a panel that had been incorrectly repaired after a tailstrike accident 7 years earlier. This caused a rapid decompression of the aircraft, bringing down the ceiling around the rear lavatories, damaging the unpressurized fuselage aft of the bulkhead, unseating the vertical stabilizer, and severing all four hydraulic lines. A photograph taken from the ground confirmed that the vertical stabilizer was missing.\par
The pilots set their transponder to broadcast a distress signal. Tokyo Area Control Center directed the aircraft to descend and follow emergency landing vectors. Because of control problems, Captain Takahama requested a vector to Haneda, declining ATC's suggestion to divert to Nagoya Airport.\par

Hydraulic fluid completely drained away through the rupture. With total loss of hydraulic control and non-functional control surfaces, plus the lack of stabilizing influence from the vertical stabilizer, the aircraft began up and down oscillation in a phugoid cycle. In response, the pilots exerted efforts to establish stability using differential engine thrust. Further measures to exert control, such as lowering the landing gear and flaps, interfered with control by throttle, and the aircrew's ability to control the aircraft deteriorated.
The aircraft after rapid decompression, with its vertical stabilizer missing.\par
Upon descending to 13,500 feet (4100 m), the pilots reported an uncontrollable aircraft. Heading over the Izu Peninsula the pilots turned towards the Pacific Ocean, then back towards the shore; they descended below 7,000 feet (2100 m) before returning to a climb. The aircraft reached 13,000 feet (4000 m) before entering an uncontrollable descent into the mountains and disappearing from radar at 6:56 p.m. at 6,800 feet (2100 m). In the final moments, the wing clipped a mountain ridge. During a subsequent rapid plunge, the plane then slammed into a second ridge, then flipped and landed on its back.\par

The aircraft's crash point, at an elevation of 1,565 metres (5,135 ft), is located in Sector 76, State Forest, 3577 Aza Hontani, Ouaza Narahara, Ueno Village, Tano District, Gunma Prefecture. The east-west ridge is about 2.5 kilometres (8,200 ft) north north west of Mount Mikuni. Ed Magnuson of Time magazine said that the area where the aircraft crashed was referred to as the "Tibet" of Gunma Prefecture. The elapsed time from the bulkhead failure to the crash was 32 minutes.\par

The Japanese public's confidence in Japan Airlines took a dramatic downturn in the wake of the disaster, with passenger numbers on domestic routes dropping by one third. Rumors persisted that Boeing had admitted fault to cover up shortcomings in the airline's inspection procedures, thus protecting the reputation of a major customer. In the months after the crash, domestic traffic decreased by as much as 25\%. In 1986, for the first time in a decade, fewer passengers boarded JAL's overseas flights during the New Year period than the previous year. Some of them considered switching to All Nippon Airways as a safer alternative.\par

JAL paid \textyen780 million (US\$7.6 million) to the victims' relatives in the form of "condolence money" without admitting liability. JAL president, Yasumoto Takagi, resigned. In the aftermath of the incident, Hiroo Tominaga, a JAL maintenance manager, killed himself to atone for the incident, while Susumu Tajima, an engineer who had inspected and cleared the aircraft as flightworthy, committed suicide due to difficulties at work.\par

In compliance with standard procedures, Japan Airlines dropped the flight number 123 for their Haneda-Itami routes, changing it to Flight 121 and Flight 127 on September 1, 1985. While Boeing 747s were still used on the same route operating with the new flight numbers in the years following the crash, they were replaced by the Boeing 767 or Boeing 777 in the mid-1990s. The 747s continued serving JAL until their 2011 retirement. March 2 of the same year saw the retirement of the airline's final two 747s, which were -400 series.\par

In 2009, stairs with a handrail were installed to facilitate visitors' access to the crash site. Japan Transport Minister Seiji Maehara visited the site on August 12, 2010, to pray for the victims. Families of the victims, together with local volunteer groups, hold an annual memorial gathering every August 12 near the crash site in Gunma Prefecture.\par

The crash led to the 2006 opening of the Safety Promotion Center, which is located in the Daini Sogo Building in the grounds of Haneda Airport. This center was created for training purposes to alert employees to the importance of airline safety and their personal responsibility to ensure safety. The center has displays regarding aviation safety, the history of the crash, and selected pieces of the aircraft and passenger effects (including handwritten farewell notes). It is open to the public by appointment made two months prior to the visit.\par

The captain's daughter, Yoko Takahama, who was a high school student at the time of the crash, went on to become a flight attendant for Japan Airlines. Diana Yukawa, who was born after the crash, and her older sister Cassie, were the daughters of English ballet dancer Susanne Bayly and married Japanese banker Akihisa Yukawa. Yukawa died in the crash, and Bayly received a £340,000 settlement to sign papers effectively disinheriting her daughters and to remain silent, preventing embarrassment to Yukawa’s family. The sisters received an undisclosed payout from the airline in 2002. 

\subsection{The Charkhi Dadri mid-air collision in 1996}
On 12 November 1996, Saudi Arabian Airlines Flight 763, a Boeing 747 en route from Delhi, India, to Dhahran, Saudi Arabia, and Kazakhstan Airlines Flight 1907, an Ilyushin Il-76 en route from Chimkent, Kazakhstan, to Delhi, collided over the village of Charkhi Dadri, around 100 km (62 mi) west of Delhi. The crash killed all 349 people on board both planes, making it the world's deadliest mid-air collision and the deadliest aviation accident to occur in India.\par

The Saudi Arabian Airlines (Saudia) Boeing 747-168B, registration HZ-AIH, was flying the first leg of a scheduled international Delhi–Dhahran–Jeddah passenger service as Flight SVA763 with 312 occupants on board; the Kazakhstan Airlines Ilyushin Il-76TD, registration UN-76435, was on a charter service from Chimkent to Delhi as KZA1907. SVA763 departed Delhi at 18:32 local time (13:02 UTC). KZA1907 was, at the same time, descending to land at Delhi. Both flights were controlled by approach controller VK Dutta. The crew of SVA763 consisted of Captain Khalid Al Shubaily, First Officer Nazir Khan, and Flight Engineer Edris.On KZA1907, Gennadi Cherepanov served as the pilot and Egor Repp served as the radio operator.\par

KZA1907 was cleared to descend to 15,000 feet (4,600 m) when it was 74 nautical miles (137 km) from the beacon of the destination airport while SVA763, travelling on the same airway as KZA1907 but in the opposite direction, was cleared to climb to 14,000 feet (4,300 m). About eight minutes later, around 18:40, KZA1907 reported having reached its assigned altitude of 15,000 feet (4,600 m) but it was actually lower, at 14,500 feet (4,400 m), and still descending. At this time, Dutta advised the flight, "Identified traffic 12 o'clock, reciprocal Saudia Boeing 747, 10 nautical miles (19 km). Report in sight."\par

When the controller called KZA1907 again, he received no reply. He warned of the other flight's distance, but it was too late. The two aircraft had collided, the tail of KZA1907 cutting through SVA763's left wing and horizontal stabiliser. The crippled Boeing quickly lost control and went into a rapidly descending spiral with fire trailing from the wing. The Boeing broke up before crashing into the ground at 1,135 km/h (705 mph). The Ilyushin remained structurally intact as it went in a steady but rapid and uncontrolled descent until it crashed in a field. Rescuers discovered four critically injured passengers from the Ilyushin, but they all died soon afterwards. Two passengers from the Saudia flight survived the crash, still strapped to their seats, only to die of internal injuries soon after. In the end, all 312 people on board SVA763 and all 37 people on KZA1907 were killed.\par

Captain Timothy J. Place, a pilot for the United States Air Force, was the sole eyewitness to the event. He was making an initial approach in a Lockheed C-141B Starlifter when he saw that "a large cloud lit up with an orange glow".\par

The collision took place about 100 kilometres (60 mi) west of Delhi. The wreckage of the Saudi aircraft crashed near Dhani village, Bhiwani District, Haryana. The wreckage of the Kazakh aircraft hit the ground near Birohar village, Rohtak District, Haryana. This was the first mid-air collision between two commercial aircraft since the Dniprodzerzhynsk mid-air collision in 1979; it was succeeded by the mid-air collision between a Bashkirian Airlines Tupolev Tu-154M and a DHL Boeing 757 over Germany in July 2002 and then by the mid-air collision between a Gol Boeing 737 and an ExcelAire Embraer Legacy over Amazonia in September 2006.\par 
The crash was investigated by the Lahoti Commission, headed by then-Delhi High Court judge Ramesh Chandra Lahoti. Depositions were taken from the Air Traffic Controllers Guild and the two airlines. The flight data recorders were decoded by Kazakhstan Airlines and Saudia under the supervision of air crash investigators in Moscow and Farnborough, England, respectively. The ultimate cause was held to be the failure of Kazakhstan Airlines Flight 1907's pilot to follow ATC instructions, whether due to cloud turbulence or due to communication problems.\par

The commission determined that the accident had been the fault of the Kazakh Il-76 commander, who (according to FDR evidence) had descended from the assigned altitude of 15,000 to 14,500 feet (4,600 to 4,400 m) and subsequently 14,000 feet (4,300 m) and even lower. The report ascribed the cause of this serious breach in operating procedure to the lack of English language skills on the part of the Kazakh aircraft pilots; they were relying entirely on their radio operator for communications with the ATC. The radio operator did not have his own flight instrumentation but had to look over the pilots' shoulders for a reading. Kazakh officials stated that the aircraft had descended while their pilots were fighting turbulence inside a bank of cumulus clouds.\par

Indian air controllers also complained that the Kazakh pilots sometimes confused their calculations because they are accustomed to using metre altitudes and kilometre distances, while most other countries use feet and nautical miles respectively.\par

Just a few seconds from impact, the Kazakh plane climbed slightly and the two planes collided. This was because the radio operator of Kazakhstan 1907 discovered only then that they were not at 15,000 feet and asked the pilot to climb. The captain gave orders for full throttle, and the plane climbed, only to hit the oncoming Saudi plane. The tail of the Kazakh plane clipped the left wing of the Saudi jet, severing both parts from their respective planes. Had the Kazakh pilots not climbed slightly, it is likely that they would have passed under the Saudi plane.\par

The recorder of the Saudi plane revealed the pilots recited the prayer that is required, according to Islamic law, when one faces death. The counsel for the ATC Guild denied the presence of turbulence, quoting meteorological reports, but did state that the collision occurred inside a cloud. This was substantiated by the affidavit of Capt. Place, who was the commander of the aforementioned Lockheed C-141B Starlifter, which was flying into New Delhi at the time of the crash. The members of his crew filed similar affidavits.\par

Furthermore, Indira Gandhi International Airport did not have secondary surveillance radar, which provides extra information, such as the aircraft's identity and altitude, by reading transponder signals; instead the airport had primary radar, which produces readings of distance and bearing, but not altitude. In addition, the civilian airspace around New Delhi had one corridor for departures and arrivals. Most areas separate departures and arrivals into separate corridors. The airspace had one civilian corridor because much of the airspace was taken by the Indian Air Force. Due to the crash, the air-crash investigation report recommended changes to air-traffic procedures and infrastructure in New Delhi's air-space:
\begin{enumerate}
\item Separation of inbound and outbound aircraft through the creation of 'air corridors'
\item Installation of a secondary air-traffic control radar for aircraft altitude data
\item Mandatory collision avoidance equipment on commercial aircraft operating in Indian airspace
\item Reduction of the airspace over New Delhi that was formerly under exclusive control of the Indian Air Force
\end{enumerate}
\subsection{The missing flight MH370}
Malaysia Airlines Flight 370 was a scheduled international passenger flight operated by Malaysia Airlines that disappeared on 8 March 2014 while flying from Kuala Lumpur International Airport, Malaysia, to its destination, Beijing Capital International Airport in China. Commonly referred to as "MH370", "Flight 370" or "Flight MH370", the flight was also marketed as China Southern Airlines Flight 748 (CZ748/CSN748) through a codeshare. The crew of the Boeing 777-200ER aircraft last communicated with air traffic control (ATC) around 38 minutes after takeoff when the flight was over the South China Sea. The aircraft was lost from ATC radar screens minutes later, but was tracked by military radar for another hour, deviating westwards from its planned flight path, crossing the Malay Peninsula and Andaman Sea. It left radar range 200 nautical miles (370 km) northwest of Penang Island in northwestern Malaysia. With all 227 passengers and 12 crew aboard presumed dead, the disappearance of Flight 370 was the deadliest incident involving a Boeing 777 and the deadliest incident of Malaysia Airlines' history, until it was surpassed in both regards by Malaysia Airlines Flight 17 four months later. The combined loss caused significant financial problems for Malaysia Airlines, which was renationalised by the Malaysian government in late 2014.\par

The search for the missing airplane, which became the most costly of aviation history, emphasized initially the South China and Andaman seas, before analysis of the aircraft's automated communications with an Inmarsat satellite identified a possible crash site somewhere in the southern Indian Ocean. The lack of official information in the days immediately after the disappearance prompted fierce criticism from the Chinese public, particularly from relatives of the passengers; most on board Flight 370 were of Chinese origin. Several pieces of marine debris confirmed to be from the aircraft were washed ashore in the western Indian Ocean during 2015 and 2016. After a three-year search across 120,000 km2 (46,000 sq mi) of ocean failed to locate the aircraft, the Joint Agency Coordination Centre heading the operation suspended their activities in January 2017. A second search launched in January 2018 by the private contractor Ocean Infinity also ended without success after six months.\par

The disappearance of Flight 370 has been dubbed one of the greatest aviation mysteries of all time. Relying mostly on analysis of data from the Inmarsat satellite with which the aircraft last communicated, the Australian Transport Safety Bureau proposed initially that a hypoxia event was the most likely cause given the available evidence, although there has not been any consensus concerning this theory among investigators. At various stages of the investigation, possible hijacking scenarios were considered, including crew involvement, and suspicion of the airplane's cargo manifest; many unofficial theories have also been proposed by the media. The Malaysian Ministry of Transport's final report from July 2018 was inconclusive, but highlighted Malaysian air traffic controllers' failures to attempt to communicate with the aircraft shortly after its disappearance. In the absence of a definitive cause of the disappearance, safety recommendations and regulations of the air transport industry, citing Flight 370, have been intended mostly to prevent a repetition of the circumstances associated with the loss. These include increased battery life on underwater locator beacons, lengthening of recording times on flight data recorders and cockpit voice recorders, and new standards for aircraft position reporting over open ocean. \par

\section{The Flow of This Investigation}
I present the investigation into the \textit{challenges in reducing accidents and incidents in aviation industry} as follows:
\begin{enumerate}
		\item \textit{Chapter One} illustrates the definition of accidents and incidents, the cause of these accidents and incidents and the examples of accidents that shape the aviation regulation worldwide.
		\item \textit{Chapter Two} illustrates the context of safety standards in commercial aviation and the regulatory bodies that delve into the investigation of accidents and incidents in aviation industries.
		\item \textit{Chapter Three} illustrates the state of the art method to assess and model the risk in aviation and the outlining problems presented in these models.
		\item \textit{Chapter Four} illustrates the challenges and method to reduce the aviation accidents and incidents caused by human error.
		\item \textit{Chapter Five} illustrates the state of the art technologies and systems to reduce the aviation accidents and incidents.
		\item \textit{Chapter Six} concludes this report.
\end{enumerate}

\end{document}
