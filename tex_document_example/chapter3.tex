\documentclass[a4paper, 10pt]{article}
\title{Chapter Three: The Overview of Mathematical Models of Risk in Aviation}
\author{Rabiah binti Tukiman}
\usepackage{textcomp}
\begin{document}
\maketitle
\section{Introduction}
Many developments in aviation are initiated as a direct
result from aircraft accidents. One of them is development of
risk and safety methods/models at beginning of 1960’s. As a
reaction on accidents, first causal methods/models are
developed with aim to find out their main causes in order to
prevent further accidents. In the same time, collision risk
methods/models appeared with proactive role in redesigning
the air traffic system in order to safely accommodate increasing
traffic demand. Since 1970’s, aviation community become
more concerned in a human roles in accidents, resulting in
development of Human factor errors methods/models. Latter
on, during 1990’s, airports appear to be a bottleneck of an air
traffic system, so the general public become aware of severity
of accidents in airports vicinity and their influence on
surrounding inhabitants and environment. Increased awareness
was resulting in development of Third-party risk
methods/models. Causal methods/models for risk and safety
assessment of aircraft and ATC/ATM operations, in particular,
deals with failures of particular technical systems and components resulting in the aircraft crash or collision. The
failures can be due to many interrelated causes and happen
either in the aircraft or at ATC/ATM. Collision risk
methods/models are dealing with assessment of the risk of
aircraft collision while airborne and/or on the ground due to
deterioration of ATC/ATM separation rules. Human factor
error methods/models deals with risk and safety assessment of
air traffic incidents and accidents due to human error. Third party risk methods/models consider the risk assessment for
people on the ground, who might be affected by the aircraft
crash.\par

The main criterion for selection of particular
methods/models has been the authors’ judgment about their
both theoretical importance and practical contribution
(although authors were well aware of existence of many other
models and similar previous studies). Also, authors’ are
focusing on proactive modeling approach, i.e. on
methods/models which are attempting to anticipate problems
before accidents occur, presenting their purpose and related
problems.\par

\section{Causal Methods/Models for the Risk and Safety Assessment of Aircraft and ATC/ATM Operations}

Causal methods/models of assessment of risk and safety of
aircraft and ATM/ATC operations establish the theoretical
framework of causes that might lead to aircraft accidents.
These methods/models can be qualitative or quantitative. The
former provide a diagrammatic or hierarchical description of
the factors that might cause accidents. They are useful for
improving understanding of causes of accidents and proposing
preventive interventions. The later estimate the probability of
occurrence of each cause and hence estimate the risk of
accident. They might be restricted to pure statistical analysis
based on the available data or combine these data with expert
judgment on the accident causes. In addition, they can estimate
the relative benefits of different interventions aiming at
preventing accidents in the future. Some of the
methods/models are as follows:
\begin{enumerate}
		\item Fault Tree Analysis (FTA) is method developed by
				Bell Telephone Laboratories, US in 1961 and has been used
				for analyzing events or combinations of events that might lead
				to a hazard or an event with serious consequences. Usually, the
				analysis has been carried out using a fault tree with several
				paths representing different combinations of instant-direct and
				intermediate causes described with logical operators (“and” and
				“or”). At the top of the tree there is a hazard event or a serious
				consequence. Then, for a given tree the minimum cut set has
				been determined, i.e., the minimal set of failures of which if all
				happen causes the top event to happen too. One fault tree might
				have several minimal cut sets, and if only one happens, the top
				event also happens. The probability of occurrence of given
				minimum cut sets is equivalent to the product of probabilities
				of occurrence of each event within the set. Consequently, the
				probability of the occurrence of the top event is equal to the
				sum of probabilities of particular minimum cut sets. The
				method has been frequently applied (as the best recommended)
				to assessment of risk and safety as well as reliability of the
				aircraft and ATC/ATM computer (hardware) components;
		\item Common Cause Analysis (CCA) is the method, which
				can be used for identifying sequences of events leading to an
				aircraft accident. In particular, the method appears useful to
				extract common causes of several aircraft accidents. For such a
				purpose, it “divides” the aircraft into “zones” implying that the
				system and components in each zone are ultimately
				independent. Consequently, it is possible to identify the
				common causes of failures of particular components of such
				independent systems. The NASA has used this method for a
				long time (since 1987) although the method itself is probably
				older then 1975. In addition, it has been recommended for
				assessment of the risk of failures of aircraft systems and
				equipment;
		\item Event Tree Analyses (ETA) method is developed in
				1980 and is used for modeling sequences of events arising from
				a single hazard and consequently describe seriousness of the
				outcomes from these events. The hierarchy of presenting a
				hazard, the sequence of events causing failures of the system
				components, and their state in terms of functioning and failure,
				represent the core of the method. Consequently, a tree with
				branches of events and functioning and failing components
				displays probabilities of failures along particular branches.
				These in combination with the probability of the hazardous
				event enable quantification of the probability of the system or
				component failure. This method has shown it is applicable in
				combination with FTA (Fault Tree Analysis) for almost all
				technical systems including the aircraft and ATC/ATM
				components. Bow-Tie Analysis presents a combination of ETA
				and FTA. Origins are from 1970’s and 1980s, but since 1999
				have been popularized as a structured approach for risk
				analysis;
		\item TOPAZ accident risk assessment methodology is a
				complex method that uses scenario analysis and a Monte Carlo
				simulation technique for assessment of the risk and safety of
				ATC/ATM operations modeled as a Petri Nets. It has been
				developed by NLR (The Netherlands National Aerospace
				Laboratory) during the 1990’s. The method addresses all types
				of system safety issues such as technical/technological,
				organizational, environmental, and human-related and other
				hazards and their combinations. Risk and safety assessment is
				performed through few steps enabling identification of safety
				bottlenecks. The method has been widely applied to risk
				assessment of ATC/ATM operations;
		\item Bayesian Belief Networks (BBN) is a method based
				on probability theory, which has been developed to improve
				understanding of the impacts of different causes on the risk of
				aircraft accidents (originating from mid of 1980’s, applied in
				aviation filed at beginning of 2000’s). The method is supposed
				to capture the wide range of failures of aircraft systems both
				qualitatively and quantitatively and thus provide rather
				objective and unambiguous information on the state of system
				safety relevant for the managerial decisions. The
				method has been applied as a decision-support tool to calculate
				effects of specific changes to the aviation system on the overall
				risk as well as support in developing a proactive policy by
				providing an insight into the effects of anticipated system
				changes on risk.
\end{enumerate} \par

Increasingly interesting causal methods/models have
mainly been used for:
\begin{enumerate}
	\item better understanding of effects of
different influencing factors on level of risk; 
	\item evaluation of
overall risk, risk communication, and cost-benefit analysis of
new technologies; 
	\item training of aviation staff and
identification of system components that could be improved;
and 
	\item iv) identifying “critical” causes of the aircraft accident as
well as measures for reducing risk. For example, in order to
decide which measures for risk reduction should be adopted;
regulators and safety managers need an understanding of
causes of accidents and an ability to evaluate benefits of
various interventions. These methods/models can support these
decisions. All mentioned methods/models are quantitative
except the CCA. Related to risk types given in Section II, it
could be mentioned that FTA, ETA and CCA are generally
used to determine “statistical risk” of occurrence of an accident
or failures, while Bow-Ties, TOPAZ and BBN - “predicted
risk” of system changes such as introduction of new
technologies, procedures, operations, etc.
\end{enumerate}\par

The causal methods/models are data driven and highly
dependant in their quality on the one hand and the expert
judgment about combinations of particular causal factors of the
air traffic accidents on the other. Quantification of these
methods/models has appeared extremely difficult and time
consuming mainly due to the complexity of combinations of
causal factors leading to possible accidents. In addition,
calculation of probabilities and conditional probabilities in
situations where dependencies between particular causal
factors have not completely been known further complicates
quantification of the methods/models. As well, one important
problem has been the cumulative nature of these
methods/models, which could make assessment of particular
probabilities difficult due to the large number of causal factors
and their combinations. Consequently, in some cases it has
been rather difficult to express results from these
methods/models in a transparent and comprehensible way.\par

It is desirable that causal methods/models posses some
predictive capabilities, i.e., not only predicting the risk level
and causal breakdown but also indicating their variations
within changing input assumptions. Such capability would
enable these methods/models to reflect better the already
adopted safety measures as well as eventual benefits of further
improvements. In addition, they should be able to assess the
safety bottleneck in the existing system, i.e., its most
vulnerable component. Due to the very complex and
demanding modeling process; modular development could
eventually be a compromise solution for these methods/models.
This could imply starting with official statistics on air traffic
accidents, and later on, allowing integration of particular
modules into more complex networks. In addition, these
methods/models could be developed specifically for airports,
ATC/ATM, and airlines as components of the civil aviation
system.

\section{Collision Risk Methods/Model}

One of the principal matters of concern in the daily
operation of civil aviation is preventing conflicts between
aircraft either while airborne or on the ground, which might
escalate to collision. Although aircraft collisions have actually
been very rare events contributing to a very small proportion of
the total fatalities, they have always caused relatively strong
impact mainly due to relatively large number of fatalities per
single event and complete destruction of the aircraft involved.
In general, separating aircraft using space and time separation
standards (minima) has prevented conflicts and collisions.
However, due to reduction of this separation in order to
increase airspace capacity and thus cope with growing air
transport demand, assessment of the risk of conflicts and
collisions under such conditions has been investigated using
several important methods/models as follows:
\begin{enumerate}
		\item The Reich-Marks model is developed in early 1960’s
				by Royal Aircraft Establishment, UK. It is based on the
				assumption that there are random deviations of both aircraft
				positions and speeds from the expected.\par

				The model was developed to estimate the collision risk for
				flights over the North Atlantic and consequently to specify
				appropriate separation rules for the flight trajectories. The
				model computed the probability of aircraft proximity and the
				conditional probability of collision given the proximity.
				Aircraft were represented as three-dimensional boxes, i.e.,
				rectangular parallelepipeds, of given length, width and height
				reflecting the ATC/ATM minimum separation rules. The
				collision might occur whenever any two boxes intersected. As
				well, when one aircraft was represented as the dimensionless
				point, conflict occurred when the point entered the box. In such
				a context the collision risk with the vertical, lateral and
				longitudinal neighbor could be determined independently of
				each other bearing in mind that the position errors of boxes and
				points representing the aircraft along their tracks were random
				variables with zero mean and given standard deviations.
				Consequently, the prescribed lateral distance between aircraft
				could be specified with given probability of violation reflecting
				the acceptable collision risk;
		\item The Machol-Reich model was developed after the
				ICAO had established the NAT SPG (North Atlantic System
				Planning Group) in 1966 with the idea of creating the Reich-
				Marks model as the workable tool as well as increase of
				airspace capacity. The modified model using actual data for the
				position error (collected for about 14000 flights) enabled
				prediction with moderate confidence of each of the vertical,
				horizontal and longitudinal collision risks. Consequently, the
				ICAO NAT SPG has adopted the threshold for risk of collision
				of two aircraft due to the loss of planned separation;
		\item The geometric conflict models are similar to the
				intersection models. In these models (developed in 1990’s) the
				speed of any two aircraft is constant, but their initial three-
				dimensional positions are random. Based on extrapolating their
				positions in time, it is possible to geometrically describe the set
				of initial locations that eventually lead to a conflict. The
				conflict occurs when two aircraft are closer than the prescribed
				separation rules. After integrating the probability density of the
				initial aircraft positions over the conflicting region, the conflict
				probability can be estimated;
		\item Generalized Reich model was developed by removing
				restrictive assumptions of Reich model based on the fact that
				Reich model does not adequately cover some real air traffic
				situations. The model was based on the hybrid-state Markov
				processes, aiming to cover a larger variety of air traffic
				situations. The resulting collision risk equals the probability of
				collision between two aircraft. Such a generalized collision
				model was developed during 1990’s and has been used as part
				of the TOPAZ methodology (mentioned in Section II, A).
\end{enumerate}\par
The main driving force for developing collision risk
methods/models during the 1960’s was the need for increasing airspace capacity over Atlantic through decreasing aircraft
separation minima. The methods/models were expected to
show if reduction of separation and spacing between the flight
tracks would be sufficiently safe, i.e., determine the appropriate
spacing between tracks guaranteeing a given level of safety.
The collision risk methods/models have gradually been
developed from Marks, Reich and Machol to the latest versions
used in TOPAZ methodology. The main purpose has always
remained to support decision-making processes during system
planning and development through evaluation of the risk and
safety of proposed changes (either in the existing or new
system). Methods/models from this category, according to risk
classification from Section II, generally provide an assessment
of “predicted risk” and implicitly “real risk to an individual”
due to the fact that collisions are usually leading to fatalities.\par

Despite the collision risk methods/models having been
successfully used for a long time (more than 40 years), some
problems, which could make their further use even more
complex have continued to exist as follows:
\begin{enumerate}
		\item Complexity and cost of collecting the enormous amount
of data on aircraft three-dimensional positions necessary to
define the related statistical distributions;
\item Inherent complexity of the generic collision risk
method/model as the result of the modeling approach (closer to
the reality). New versions of these methods/models such as
those used in TOPAZ are even more complex because they
embrace more details when calculating risks, such as possible
failure of some technical systems (engine, avionics, etc.) or
flight crew awareness or fatigue; and cover complex
relationships between elements of the system (flight crew,
aircraft, ATC/ATM system, other aircraft, etc.);
\item Inherent danger of misunderstanding or no understanding
from the average user’s point of view mainly due to
complexity. This requires of the specialists a long and costly
familiarization time;
\item The lack of risk-predicting capability with high degree of
confidence and bias and uncertainty of the obtained results.
Additional time and expertise for calculation of the credible
risk intervals are needed;
\item Relying on expert judgment in cases where historical
data are not available, or when their collection is very
expensive: the experts are used for setting up the value of
parameters, value and dispersion of the random variables, and
the dependence between variables. In such contexts, there is
always the problem of engaging credible experts, especially in
cases involving new system concepts;
\item Complexity in validation particularly of new system
concepts. In cases of non-existent systems, the ICAO has
recommended comparison with the reference system and
evaluating risk against its given threshold value.\par
\end{enumerate}
Regarding the purpose and existing structure, certain
compromise in terms of obtaining some kind of balance
between complexity and usability (due to enormous amount of
input data and high level of the necessary expertise) might be
recommended. Additional recommendations would be
development of the method/models for specific purposes such
as collision risk assessment in the en-route and terminal
airspace or at the airport as well as devotion to their use at local
level particularly while assessing the effects of new equipment
on the collision risk. Finally, these methods/models should
have better predictive capability because their usage will be
more and more related to collision risk assessment when new
systems, procedures, concepts and operations are introduced.

\section{Human Factor Error Methods/Model}

Investigation of causes of particular air traffic accidents has
identified “human error” as one of the most frequent causes. Human error is considered as an incorrect execution of a
particular task, which as an event, triggers a series of
consecutive errors in execution of other tasks, finally resulting
in serious consequences. Therefore, monitoring and modeling of human errors in the
aircraft and ATC/ATM operations aiming at discovering and
preventing them have always been high on the research agenda
of both academics and practitioners dealing with civil aviation.
Consequently, many methods for detection and prevention of
				“human errors” have been developed; some of them are:
\begin{enumerate}
		\item HAZOP (Hazard and Operability) method (developed
				in early 1970’s) aims at discovering potential hazards,
				operability problems, and possible deviations of the actual from
				the system intended operational conditions (states) including
				estimating the probability of escalation into a serious event.
				The method was intended to deal with human errors in
				complex technical systems such as chemical and nuclear plants
				having human operator in their control loop. Later on, the UK
				NATS (National Air Traffic Service) applied the method to
				different aspects of planning and assessing hazard in operation
				of the national ATC/ATM, particularly for identification of
				hazards due to human failures that might develop into risk of
				air traffic accidents (HAZOP can provide input to FTA and
				ETA, mentioned in Section III, A);
		\item HEART (Human Error Assessment and Reduction
				Techniques) was developed in 1985 for identifying and
				quantifying errors in an operator’s task. It simultaneously
				considers particular ergonomic and other environmental
				factors, which might compromise the required operator’s
				performance. The impact of a particular (each) factor on the
				operator’s error while performing particular tasks can be
				quantified. Then the probability of error in executing a given
				task (or a series of tasks) can be estimated. The method has
				been applied by the UK NATS in combination with other
				methods for identification of the human errors in ATC/ATM;
		\item TRACER-Lite (Technique for the Retrospective
				Analysis of Cognitive Errors) was developed in 1999 by
				NATS, for predicting human errors and deriving error
				prevention measures in ATC/ATM. The method is
				retrospective, i.e., it is used for classifying types of errors
				contributing to the air traffic incidents, which have already
				happened. The method has a modular structure with three
				modules: the context; the error discovery; and the error
				recovery. Hierarchical Task Analysis enabling identification of
				the “set of critical” tasks, critically influencing safety, usually
				classifies the human errors;
		\item HERA (Human Error in ATM) is the retrospective
				method providing insight into ATC/ATM controllers’ cognitive
				processes while dealing with air traffic incidents (developed at
				EUROCONTROL at beginning of 2000’s). The method
				consists of two parts: a retrospective part for the incident
				analysis; and a prospective part using the information collected
				on the assessment of probability of human error in cases of
				compromised safety. Consequently, the method enables better
				understanding of the constraints and conditions under which
				ATC/ATM controllers operate. These conditions are important
				for understanding ATC/ATM controllers’ incompliance with
				existing procedures and skill-related errors;
		\item HFACS (Human Factor Analysis and Classification
				System) is method developed at beginning of 2000’s in USA,
				as a system to categorize latent and immediate causal factors
				that have been identified in aviation accidents. It is based on
				analysis of hundreds of aviation accident reports and main purpose is to provide a framework for accident investigations
				and to serve as a tool for accident trends assessment. HFACS
				uses four levels of failure: i) unsafe acts; ii) preconditions for
				unsafe acts; iii) unsafe supervision and iv) organizational or
				cultural influences. The method is very promising for analysis
				of air traffic controller errors and failures in ATC/ATM and is
				effective for understanding the antecedents of operational
				errors for air traffic safety analysis.
\end{enumerate}\par

The methods/models dealing with human factor errors in
civil aviation have been developed to identify and eventually
prevent errors (particularly of aircraft crew and ATC/ATM
controllers), which could cause aircraft incidents and accidents.
In addition, these models have investigated factors from the
operational environment, which could cause errors, as well as
calculating the probability of making errors in performing
given activities. Consequently, it will be expected that they will
be applied to both operational and design stages of developing
aviation systems. Specific types of methods/models have given
insight into the cognitive processes of the ATC/ATM
controllers operating in the incidental situations, analyzed these
situations, and calculated probability of making errors. In
addition, these methods/models have possessed some ability
for predicting errors and specifying the error reduction
measures. According to risk types in Section II, those
methods/models are mostly intended to determine “statistical”
and “predicted” risk for given probability of error.\par

Human factor errors methods/models posses some
shortcomings, which might compromise their more efficient
and effective application to the ATC/ATM as follows:
\begin{enumerate}
		\item Most activities in ATC/ATM and in particular, factors
influencing human operator performance and possible errors
have usually been considered in isolation, i.e., independently
on each other; in many cases the quantitative information has
exclusively relied on expert judgment;
\item Only specialists in ”human factors” have been able to
use these methods/models efficiently and effectively; i.e., it has
been time consuming and almost impossible to apply these
methods/models in an operational environment without
specialists;
\item The methods/models have been constrained exclusively
to the operational processes and activities in the ATC/ATM.
\end{enumerate}\par

Human factor error methods/models with necessary
modifications should be applicable to new technologies and
systems in ATC/ATM for identifying human errors at all levels
of system functioning and they should be able to generate
measures for error prevention and/or reduction already at the
design stage. For such purposes, they will have to be able to
handle careful specification of activities and tasks throughout
the system in a way, which will not be highly if not crucially
dependent on the highly specialized staff.

\section{Third-Party Risk Method/Models}
Third-party risk implies risk if an individual on the ground
to be killed by crashing aircraft. In such a case, the accident is
called a “groundling accident” or “groundling crash” and the
fatality a “groundling fatality”. Since most air traffic accidents
(about 70\%) happen around airports, the
concept and assessment of third-party risk has been mainly
focused on areas around airports. In a given context, the basic
assumption has been that risk always exists, cannot be reduced
to zero and should be predictable, transparent, and controllable,
as well as quantifiable and measurable. Modeling of third-party
risk has shown promise in resolving these problems including
setting up thresholds for acceptable risk around airports. Three cases of assessment of the third-party risk are
illustrated as follows:
\begin{enumerate}
		\item \textit{\textbf{USA case}}:  generally implies assessment of the risk
				an individual is exposed to when at some distance from a given
				airport during the period of a year. For such a purpose, relevant
				statistics on fatalities from official sources have been collected
				and the prospective number of ground fatalities estimated. The
				estimation has been carried out by multiplying two independent
				variables – the number of crashes around airports and the
				number of fatalities per individual crash. The model has shown
				that the probability of being killed by crashing aircraft has
				decreased more than proportionally with increasing distance
				from the airport and increased with increase in the volume of
				the airport traffic at distances up to about two miles. The model
				has not considered spatial variability of the risk due to
				changing residence locations and the aircraft flight paths
				around the airports, which might be considered as its main
				disadvantage;
		\item \textit{\textbf{Netherland case}}: this method was developed by
				the NLR, inspired by the crash of cargo aircraft in the Bijlmer
				district of Amsterdam in 1992. Method contain the following
				elements:
				\begin{enumerate}
						\item the accident probability model, which
				calculates the probability of an aircraft accident in the vicinity
				of an airport depending on the probability of an accident per
				aircraft movement and the annual volume of airport traffic; 
						\item ii)the accident location probability model, which calculates the
				probability of a given location becoming an accident scene
				depending on its position relative to airport runways and the
				incoming and outgoing aircraft trajectories; and 
						\item the
				accident effect model, which combines output from both
				previous models to calculate the probability of an accident at
				each location within the area surrounding a given airport.
				Individual and societal risks have been used as measures of
				third-party risk. After calculating the individual risks for the
				entire area around given airport, the risk contours can be
				plotted on the horizontal plane. Societal risk applies to the
				entire area around a given airport and actually exists only when
				people are actually present in the area;
				\end{enumerate}
		\item \textit{\textbf{UK case}} - has become important after Public Safety
				Zones (PSZs) were introduced in 1958. The PSZ was defined
				as an area adjacent to the end of a runway in which
				development of land had to be restricted if it would likely
				significantly increase the number of “residing, working or
				congregating people there”. In the 1997 the method for
				third-party risk assessments around airports and the proposal of
				the appropriate risk assessment criteria was developed in a
				NATS. The method was based on distinguishing aircraft
				regarding their manufacturer, country of origin, type (large,
				small, jets, turbo-props), and category (passenger, cargo),
				modeling of the aircraft crash location and the crash
				consequences both based on a limited sample, and simplified
				approach, to draw the risk contours around a given airport. In
				addition, cost-benefit analysis was applied to set up criteria for
				acceptable (tolerable) risk.
\end{enumerate}\par
The third-party methods/models have been mainly used for
decision-making and policy purposes related to airport
development and operations as follows: \textit{(a)} forecasting risk for
an individual to be killed by a crashing airplane in the vicinity
of given airports. The information has been used for comparing
the risk around airports and that around chemical or nuclear
plants;\textit{(b)} zoning around airports using individual risk contours and societal risk values, i.e. determining areas, which should be
considered dangerous for building houses or other vulnerable
infrastructure; \textit{c)} indicating changes in risk contours arising
from airport development or changes in using existing
infrastructure (changes of runways in use, arrival or departure
trajectories, etc).\par

The third-party methods/models have been permanently
improved and updated. The main problems identified during
that process have been as follows: \textit{(a)} lack of generality,
i.e., the specific method/model has been developed for the
specific airport; \textit{(b)} proactive assessment of the risk could not
be carried out due to the risk control measures being already in
place; \textit{(c)} scarcity of data on real accidents and risk exposure
around the airports in the official statistical sources; \textit{(d)}
difficulties in setting up threshold values for individual and
societal risk; if too high it might compromise the airport
operations and development; if too low, it might put
individuals at an unacceptable jeopardy.\par

Predictive capabilities and flexibility of third-party risk
methods/models will be essential to produce new (updated)
individual and societal risk estimates based on the expected
number of fatalities after introducing new technologies and
operational procedures at given airport. On the one hand these
are expected to increase airport capacity and on the other they
should decrease the accident rate in the vicinity of airports.

\end{document}
